\CTEXsetup[number={\arabic{chapter}}]{chapter}
\CTEXsetup[name={,}]{chapter}
\CTEXsetup[nameformat={\bf\heiti\xiaosan\raggedright}]{chapter}   %	Chapter章节标号格式
\CTEXsetup[titleformat={\bf\heiti\xiaosan\raggedright}]{chapter}      %	Chapter章节标题格式
\CTEXsetup[beforeskip={-8ex}]{chapter}                          
\CTEXsetup[afterskip={1ex}]{chapter}


\lstset{language=Matlab}

\appendix
\begin{appendix}
\chapter{证明}
\begin{spacing}{1.5}
%-------------------------------------------------------------------------------
%	AppendixOne正文部分
%-------------------------------------------------------------------------------
提供一个证明环境,给出一个证明示例。

下面证明命题,如果$\bm{A}$是正交矩阵,那么其特征值为$\pm1$。
\begin{proof}
设$\lambda$ 为正交矩阵的特征值,$\bm{e}$是$\bm{A}$ 属于$\lambda$ 的特征向量,即有:$\bm{A}\bm{e}=\lambda\bm{e}$,且$\bm{e}\neq0$。

上式两边取转置,$\bm{e}^{T}\bm{A}^{T}=\lambda\bm{e}^{T}$。

\text{将}上面两式相乘, $\bm{e}^{T}\bm{A}^{T}\bm{A}\bm{e}=\lambda^2\bm{e}^{T}\bm{e}$。

$\because\bm{A}$是正交矩阵

$\therefore\bm{A}^{T}\bm{A}=\bm{I}$

$\therefore\bm{e}^{T}\bm{e}=\lambda^2\bm{e}^{T}\bm{e}$

又 $\because\bm{e}\neq0$

$\therefore\bm{e}^{T}\bm{e}$ 是一个非零的数

$\therefore\lambda^2=1\Rightarrow\lambda=\pm1$
\end{proof}



\end{spacing}
%------------------------------------------------------------------------------


\chapter{MATLAB Code}

%-------------------------------------------------------------------------------
%	AppendixOne正文部分
%-------------------------------------------------------------------------------
下面是一个MATLAB程序的事例,使用了Package mcode,它能较好还原MATLAB本身的编写风格。
\begin{lstlisting}
%   The program normalizes the measurement data and compares it to the standard cosine function
data=xlsread('data_sun',1,'B3:E39');
min=[(data(1,1)+data(37,1))/2,(data(1,2)+data(37,2))/2,...
    (data(1,3)+data(37,3))/2,(data(1,4)+data(37,4))/2];
max=[data(19,1),data(19,2),data(19,3),data(19,4)];
Min=repmat(min,37,1);
Max=repmat(max,37,1);
data=(data-Min)./(Max-Min);
x=-pi/2:pi/36:pi/2;
y=cos(x);
%----------------------figure-------------------------%
figure(1);
subplot(2,2,1);
plot(x,data(:,1),'ro-');
hold on;
plot(x,y,'b-');
title('R=1.2\Omega');
axis([-2,2,0,1]);
grid on;
subplot(2,2,2);
plot(x,data(:,2),'ro-');
hold on;
plot(x,y,'b-');
title('R=1.6\Omega');
axis([-2,2,0,1]);
grid on;
subplot(2,2,3);
plot(x,data(:,2),'ro-');
hold on;
plot(x,y,'b-');
title('R=2.0\Omega');
axis([-2,2,0,1]);
grid on;
subplot(2,2,4);
plot(x,data(:,4),'ro-');
hold on;
plot(x,y,'b-');
title('R=2.4\Omega');
grid on;
axis([-2,2,0,1]);
\end{lstlisting}

%------------------------------------------------------------------------------

\end{appendix}
